\section{Witek Nowogórski}
\label{sec:Nowogorski}

\begin{figure}[htbp]
    \centering
    \includegraphics[width=0.6\textwidth]{pictures/dezodorant.jpg}
    \caption{dezodorant kielecki}
    \label{fig:kielecki}
\end{figure}

\subsection{Trochę fizyki}
\underline{Statystyka Fermiego-Diraca} – statystyka dotycząca fermionów – cząstek o spinie połówkowym, które obowiązuje zakaz Pauliego. Zgodnie z zakazem Pauliego w danym stanie kwantowym nie może znajdować się więcej niż jeden fermion. Statystyka Fermiego-Diraca oparta jest również na założeniu nierozróżnialności cząstek[1].

Jego nazwa rozkładu pochodzi nazwisk fizyków Enrico Fermiego-Paula Diraca, którzy niezależnie od siebie wyprowadzili tę zależność w 1926 roku[2][3].

Zgodnie z rozkładem Fermiego-Diraca średnia liczba cząstek w niezdegenerowanym stanie energetycznym  E dana jest przez


$\langle n\rangle = 1:(e^{\beta(E_{k}-\mu)}+1$


\subsection{Listy zakupów}
\begin{itemize}
  \item mleko
  \item masło
  \item jajka
  \item woda
  \end{itemize}
  
  \subsection{Przepis na tosta z dżemem}
\begin{enumerate}
  \item zrobić tosta (warto użyć tostera)
  \item posmarować masłem
  \item na masło nałożyć dżem 
  \item smacznego
\end{enumerate}

\subsection{Macierze}

Table~\ref{tab:multiplication_table} przykładowa macierz 4x4

\begin{table}[htbp]
\centering
\begin{tabular}{||c c c c||} 
\hline
\hline
1 & 3 & 78 & 34  \\
\hline\
5 & 4 & 1 & 11 \\
\hline
0 & 9  & 81  & 7  \\
\hline
1  & 8  & 4  & 33  \\
\hline
\hline
\end{tabular}
\caption{macierz 4x4}
\label{tab:macierz}
\end{table}
