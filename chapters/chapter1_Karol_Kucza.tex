\section{Karol Kucza}
\label{sec:kucza}

Here is a photo of a quokka (see Figure~\ref{fig:quokka}).

\begin{figure}[htbp]
    \centering
    \includegraphics[width=0.4\textwidth]{pictures/quokka.jpg}
    \caption{It's unusually cute.}
    \label{fig:quokka}
\end{figure}

\subsection{Text}
To be fair, you have to have a \underline{very high IQ} to understand Rick and Morty. The humor is extremely subtle, and without a solid grasp of theoretical physics most of the jokes will go over a typical viewer's head. There's also Rick's nihilistic outlook, which is deftly woven into his characterisation – his personal philosophy draws heavily from \textbf{Narodnaya Volya} literature, for instance. The fans understand this stuff; they have the intellectual capacity to truly appreciate the depths of these jokes, to realize that they're not just funny – they say something deep about \emph{LIFE}. As a consequence people who dislike Rick and Morty truly \emph{ARE} idiots – of course they wouldn't appreciate, for instance, the humour in Rick's existential catchphrase  \textit{``Wubba Lubba Dub Dub``}, which itself is a cryptic reference to \textbf{Turgenev's Russian epic Fathers and Sons} I'm smirking right now just imagining one of those addlepated simpletons scratching their heads in confusion as Dan Harmon's genius unfolds itself on their television screens. What fools... how I pity them. \par
And yes by the way, I DO have a Rick and Morty tattoo. And no, you cannot see it. It's for the \underline{ladies' eyes only} – and even they have to demonstrate that they're within 5 IQ points of my own (preferably lower) beforehand.

\subsection{Fruits}
\begin{itemize}
  \item an apple
  \item a pear
  \item a raspberry
  \item a watermelon
  \item a dragonfruit
\end{itemize}

\subsection{Vegetables}
\begin{enumerate}
  \item a tomato
  \item a carrot
  \item a potato
  \item a cucumber
  \item an eggplant
\end{enumerate}

\subsection{Further reading}

\begin{table}[htbp]
\centering
\begin{tabular}{||c c c c c||} 
 \hline
 \hline
 1 & 2 & 3 & 4 & 5 \\ [0.5ex] 
 \hline
 2 & 4 & 6 & 8 & 10 \\ 
 \hline
 3 & 6 & 9 & 12 & 15 \\
 \hline
 4 & 8 & 12 & 16 & 20 \\
 \hline
 5 & 10 & 15 & 20 & 25 \\
 \hline
 \hline
\end{tabular}
\caption{Multiplication table}
\label{tab:multiplication_table}
\end{table}

Table~\ref{tab:multiplication_table} is a multiplication table.



This equation is called Euler's identity: \[e^{i\pi} + 1 = 0\]

And here is the Pythagorean identity:
$ \sin^2\alpha + \cos^2\alpha = 1 $