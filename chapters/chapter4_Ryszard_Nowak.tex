\newpage
\section{Ryszard Nowak}
\label{sec:nowak}

Żałobnica Białopasa (Figure~\ref{fig:muchowka}).

\begin{figure}[h]
    \centering
    \includegraphics[width=0.8\textwidth]{pictures/muchowka.jpg}
    \caption{Gatunek muchówki z rodziny bujankowatych i podrodziny Exoprosopinae.}
    \label{fig:muchowka}
\end{figure}

\subsection{Żałobnica białopasa}
\textbf{Muchówka} o ciele długości od 8 do 11 mm. \textbf{Ubarwienie ma matowoczarne z czarnym opyleniem.} Kulista głowa cechuje się \emph{bardzo krótkim}, niewiele dłuższym od otworu gębowego ryjkiem. Oprócz łusek czarnych wzdłuż tylnej krawędzi oczu złożonych, na przedzie czoła i twarzy występują łuski złociście żółte.
\par Czułki mają człon trzeci asymetrycznie cebulowaty, wyposażony w zwieńczoną włoskiem \textit{aristę.} Za głową znajduje się kreza z włosów czarnych, w części górnej z domieszką złociście żółtych. Po bokach wierzchu tułowia, poczynając od guzów barkowych i kończąc na guzach skrzydłowych biegnie para pasów białych lub żółtawych. \underline{Śródplecze i tarczka mają długie szczecinki na brzegach.} 
\par Łuseczki skrzydłowe są brunatnobiałe z brunatnym owłosieniem na krawędziach. Przezmianki mają brunatne trzonki i białe lub żółte główki. Przed nasadą przezmianki wyrasta pęczek żółtawych i czarnych włosków. Barwa \textit{plumuli} jest białożółta. Skrzydło na większej powierzchni jest zaczernione, przy czym zaczernienie to obejmuje również wierzchołki komórek radialnej i dyskoidalnej, co odróżnia ten gatunek od drogosza żałobnego. Odwłok ma tylne krawędzie tergitów z wąskimi przepaskami białymi, z wyjątkiem tergitu piątego, który ma na tylnym brzegu rozszerzającą się ku bokom przepaskę \underline{żółtawą}.

\subsection{Inne owady z rzędu muchówek}
\begin{enumerate}
  \item Drogosz żałobny
  \item Komar Tygrysi
  \item Bąk Bydlęcy
  \item Padlinówka Cesarska
  \item Komar Egipski
\end{enumerate}

\subsection{Inne ciekawe owady}
\begin{itemize}
  \item Kowal Bezskrzydły
  \item Nartnik
  \item Straszyk łąkowy
  \item Wtyk Północnoamerykański
  \item Osadnik wielkooki
\end{itemize}

\subsection{Tabela}

Tabela~\ref{tab:systematyka} przedstawia systematykę żałobnicy białopasej.

\begin{table}[htbp]
\centering
\begin{tabular}{|ll|}
\hline
\multicolumn{2}{|l|}{Systematyka}                      \\ \hline
\multicolumn{1}{|l|}{Domena}     & eukarionty          \\
\multicolumn{1}{|l|}{Królestwo}  & zwierzęta           \\
\multicolumn{1}{|l|}{Typ}        & stawonogi           \\
\multicolumn{1}{|l|}{Gromada}    & owady               \\
\multicolumn{1}{|l|}{Rząd}       & muchówki            \\
\multicolumn{1}{|l|}{Rodzina}    & bujankowate         \\
\multicolumn{1}{|l|}{Podrodzina} & Exoprosopinae       \\
\multicolumn{1}{|l|}{Plemię}     & Villini             \\
\multicolumn{1}{|l|}{Rodzaj}     & żałobnica           \\
\multicolumn{1}{|l|}{Gatunek}    & żałobnica białopasa \\ \hline
\end{tabular}
\caption{}
\label{tab:systematyka}
\end{table}

\subsection{Wyrażenia matematyczne}
Jakaś funkcja bez związku z Żałobnicą: $f(x) = \frac{x^2-3x+7}{\sqrt{x+1}}$.
I jeszcze jakaś granica:
\begin{equation}
\lim_{x \rightarrow 0} (\frac{\sin x}{\arccos x^2}+\sqrt[3]{x+2})
\end{equation}