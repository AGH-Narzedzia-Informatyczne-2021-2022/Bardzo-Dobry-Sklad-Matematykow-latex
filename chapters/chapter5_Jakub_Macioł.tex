\newpage
\section{Jakub Macioł}
\label{sec:maciol}

To jest zdjęcie czerwonej pandy :) (Figura ~\ref{fig:red_panda}).

\begin{figure}[htbp]
    \centering
   \includegraphics[width=0.4\textwidth]{pictures/red_panda.jpg}
    \caption{Bardzo słodka czerwona panda.}
    \label{fig:red_panda}
\end{figure}

\subsection{Tekst}
Posłuchaj mnie uważnie, jutro o 19:45 masz samolot do meksyku. \textbf{Bilet wyślę Ci zaraz na e-mail}. Gdy wyjdziesz z lotniska, \textbf{pod czerwoną budką telefoniczną jest skrytka}, otwórz ją tajnym hasłem: hajduszoboszlo. W niej znajdziesz nowy dowód osobisty, 3000 pesos i kluczyki do mieszkania naprzeciwko. Od dziś nazywasz się \textit{Juan Pablo Fernandez Maria FC Barcelona Janusz Sergio Vasilii Szewczenko} i jesteś \textsc{rosyjskim imigrantem z Rumunii}. Pracujesz w zakładzie fryzjerskim 2 km od lotniska.\par 
Powodzenia, zapomnij o swoim poprzednim życiu i pod żadnym pozorem się nie wychylaj, zerwij wszystkie kontakty, nawet z obsługą klienta z Polsatu.\par

% Table~\ref{tab:random_numbers} represents random numbers. % Do czego służy \ref{}?

\subsection{Matematyka}
To jest jakieś komplikowane wyrażenie matematyczne: 
\[ax^2+bx+c=0\]

Tablica~\ref{tab:fajneliczby} przedstawia liczby od 20 do 1.
\begin{table}[htbp]
\centering
\begin{tabular}{||c c c c c||} 
\hline
\hline
20 & 19 & 18 & 17 & 16 \\
\hline
15 & 14 & 13 & 12 & 11 \\
\hline
10 & 9  & 8  & 7  & 6  \\
\hline
5  & 4  & 3  & 2  & 1 \\
\hline
\hline
\end{tabular}
\caption{Liczby od 20 do 1}
\label{tab:fajneliczby}
\end{table}

\subsection{Składniki ciasta na pizzę}
\begin{itemize}
\renewcommand\labelitemi{--}
    \item 25 g świeżych drożdży
    \item 150 ml ciepłej wody
    \item 250 g mąki pszennej
    \item 1 łyżeczka soli
    \item 1 łyżka oliwy
\end{itemize}
\subsection{Najlepsze owoce na odporność}
\begin{enumerate}
    \item Kiwi
    \item Truskawki
    \item Pomarańcze
    \item Grejpfrut
    \item Cytryna
\end{enumerate}